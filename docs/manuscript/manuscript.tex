\documentclass[]{interact}
\usepackage{epstopdf}% To incorporate .eps illustrations using PDFLaTeX, etc.
\usepackage{subfigure}% Support for small, `sub' figures and tables
%\usepackage[nolists,tablesfirst]{endfloat}% To `separate' figures and tables from text if required

\usepackage{natbib}
\bibliographystyle{chicago}
\setcitestyle{authoryear,open={(},close={)}}
\renewcommand\bibfont{\fontsize{10}{12}\selectfont}% Bibliography support using natbib.sty

\usepackage{hyperref}
\hypersetup{
    colorlinks=true,
    linkcolor=blue,
    filecolor=magenta,
    urlcolor=blue,
    citecolor=blue,
}

\usepackage{titlesec}
\titleformat*{\section}{\Large\bfseries}
\titleformat*{\subsection}{\large\bfseries}

\usepackage{endnotes}
\let\footnote=\endnote
\usepackage{etoolbox}
\patchcmd{\enoteformat}{1.8em}{0pt}{}{}

\theoremstyle{plain}% Theorem-like structures provided by amsthm.sty
\newtheorem{theorem}{Theorem}[section]
\newtheorem{lemma}[theorem]{Lemma}
\newtheorem{corollary}[theorem]{Corollary}
\newtheorem{proposition}[theorem]{Proposition}

\theoremstyle{definition}
\newtheorem{definition}[theorem]{Definition}
\newtheorem{example}[theorem]{Example}

\theoremstyle{remark}
\newtheorem{remark}{Remark}
\newtheorem{notation}{Notation}
%%%%%%%%%%%%%%%%%%%%%%%%%%%%%%%%%%%%%%%%%%%%%%%%%%

\begin{document}

\articletype{DRAFT MANUSCRIPT}

\title{Reproducible Research Practices and Barriers to Reproducible Research in Geography: Insights from a Survey}

\author{
\name{Peter Kedron\textsuperscript{a,b}\thanks{CONTACT Peter Kedron. Email: Peter.Kedron@asu.edu}, Joseph Holler\textsuperscript{c}, and Sarah Bardin\textsuperscript{a,b}}
\affil{\textsuperscript{a}School of Geographical Sciences and Urban Planning, Arizona State University, Tempe, Arizona, USA; \textsuperscript{b}Spatial Analysis Research Center (SPARC), Arizona State University, Tempe, Arizona, USA; \textsuperscript{c}Department of Geography, Middlebury College, Middlebury, Vermont, USA}
}

\maketitle

\begin{abstract}
While the number of reproduction and replication studies undertaken in the social and behavioural sciences continues to rise, such studies have not yet become commonplace in geography. 
Existing attempts to reproduce geographic research suggest that many studies cannot be fully reproduced, or are simply missing components needed to attempt a reproduction. 
Despite this suggestive evidence, we have not yet systematically assessed geographers' perceptions of reproducibility, the use of reproducible research practices across the discipline's diverse research traditions, or identified the factors that have kept geographers from conducting more reproduction studies.
This study addresses each of these questions by surveying active geographic researchers selected using probability sampling techniques from a rigorously constructed sampling frame.
We find ...
CONClUSION...

\end{abstract}

\begin{keywords}
Reproducible Research, Epistemology, Geographic Research Methods
\end{keywords}

%%%%%%%%%%%%%%%%%%%%%%%%%%%%%%%%%%%%%%%%%%%%%%%%%%
\section*{Introduction}

\newpage

%%%%%%%%%%%%%%%%%%%%%%%%%%%%%%%%%%%%%%%%%%%%%%%%%%
\section*{Assessing the Reproducibility of Research}
The reproducibility of research can be assessed directly or indirectly. 
In a direct assessment of reproducibility, an independent team(s) of researchers repeats the analytical steps of a previous study using the same data and then evaluates the consistency of the two sets of results\footnote{Insert paragraph on the difficulty of developing in using consistency criteria}.
By contrast, during an indirect assessment of reproducibility, a research team gathers information about the transparency of a study and the availability of the research components (e.g., data and code) needed to recreate that study, but does not directly attempt to recreate the results. 
Indirect assessments also often take the form of surveys in which researchers are asked about their research practices, incentives, and motivations. 

Direct and indirect reproducibility assessments present a trade off between the depth and breadth of evaluation.  
A direct assessment that carefully reproduces a study can provide insight into how a particular study was conducted and why a consistent result could, or could not, be obtained.  
However, such assessments are resource and time intensive, which makes them difficult to scale to a large number of studies. 
Constrained to a limited number of studies, direct assessments offer only suggestive evidence about the reproducibility of unevaluated studies and the prevalence of reproducible research practices. 
In contrast, comparatively less resource intensive indirect assessments that catalog reproducible research practices can be more easily scaled.
By evaluating many studies, indirect assessments provide evidence of how common reproducible research practices are and, by extension, how reproducible work in a field of study is likely to be.
The trade off being that indirect assessments do not offer as detailed a look into why a particular result in a specific instance could, or could not, be obtained.

The available body of direct and indirect reproducibility assessments currently provides 

insufficient evidence to conclusively evaluate the reproducibility of scientific research generally, or disciplinary research specifically \citep{NASEM2019}.
This knowledge gap exists in part because while available evidence consistently suggests that inadequate record keeping, opaque reporting, and unavailable research components are drivers of research irreproducibility, how common these practices are remains unclear. 
We have been unable to estimate the prevalence and impact of these practices because indirect assessments have used non-probability based sampling methods to draw data from non-representative, self-selected populations (e.g., professional association memberships).

Prior surveys have also commonly failed to systematically report the methodological details (e.g., response rate) needed to assess and address potential bias. 

Surveys of researchers working across the sciences suggest that the majority of scientists have either not tried, or tried and failed to reproduce another scientist's work \citep{baker20161, boulbes2018survey}. 
Perceptions on the cause of irreproducibility range from natural variation to fraud \citep{ranstam2000fraud, anderson2007normative, baker20161}, with an estimated 33 percent of researchers admitting to some form of questionable research practices \citep{fanelli2009many}.  
Researchers also appear to support a broad range of solutions to the challenge of irreproducibility including improved training in research design and statistics and better research mentoring and supervision \citep{baker20161}. 



\subsection*{Assessing the Reproducibility of Geographic Research}
Direct and indirect assessments of research reproducibility 

Geographic researchers are beginning to address these questions by conducting direct and indirect assessments of the reproducibility of geographic research. 
However, only a small number of such studies currently exist resulting in a similarly limited set of evidence about reprodubibility within the discipline. 


There is reason to believe that each of these factors is present in geography, but limited evidence because published reproductions have not yet become common in geography. 
The small number of studies that have attempted to assess the reproducibility of geographic research by recreating existing work suggest that many studies cannot be fully reproduced \citep{Kedron2021ssrn, nust2018, ostermann2021}, or are simply missing components needed to attempt a reproduction \citep{Kedron_VijayanRP, konkol2019, ostermann2021}.
For example, \citet{ostermann2021}'s attempt to reproduce the computational analyses of 75 papers published in the GIScience conference series found that most results could not be reproduced, or could only be reproduced with significant effort. 
A related effort to reproduce 31 papers published in the conference series of the Association of Geographic Information Laboratories in Europe \citep{Nust_AGILE_2020, Nust2021AGILE, Nust_AGILE_2022} similarly identified the inaccessibility of data, code, and information about the computational environment as key barriers to recreating results. 
\citet{konkol2019}'s attempt to reproduce of 41 geocomputational studies likewise found that 39 of the papers had coding issues that hindered reproduction and that 47 percent of the figures created using the original data and code deviated from the those published in some significant way (e.g., graphs with different curves). 

Moving past the question of whether the computational results of a study can be recreated using the same data and code, a handful of geographic researchers are using the reproduction process as a way of examining not only the execution of a prior study, but its conceptualization and inferences \citep{Kedron_MollaloRP, Kedron_SaffaryRP, Kedron_VijayanRP}. 
While this form of the direct approach to reproduction opens a new door to the critical evaluation of geographic research, it primarily provides evidence about a particular research study and the phenomenon that study investigates. 
Similarly, while computationally focused reproductions have produced useful guidelines \citep{hofer2019reproducible, wilson2021} and moved forward the development of the computational and institutional infrastructure \citep{Kedron_Holler_Bardin_Hilgendorf_2022, nust2019, nust2021,}\textbf{ADD SHAOWEN} needed to foster reproducibility in the field, they have not yet assessed the use of reproducible research practices across the disciplines diverse research traditions, or identified the factors that have barred geographers from adopting practices that foster reproducibility or from conducting more reproduction studies. 

To date only a handful studies have tried to indirectly measure the reproducibility of geographic research.
In a survey of participants from the 2016 European Geosciences Union General Assembly, \citet{konkol2019} assessed the frequency with which researchers published their work in ways that enabled computational reproducibility. 
Those authors found that only 33 percent of respondents included links to the data used in their analyses and only 12 percent provided their analytical code. 
The authors also found that only seven percent of respondents ever attempted to reproduce the work of other researchers.
For example, \citet{ostermann2017} use a literature review of volunteered geographic information research publications to assess computational reproducibility based on availability of original data, metadata, source code, or pseudocode.
Remote Sensing survey.

\subsubsection*{Research Summary to Transition}
The evidence available from direct and indirect assessments of reproducibility suggests that studies are often not reproducible because researchers do provide the information or materials others need to recreate their work. 



However, may of our indirect assessment are on shaky methodological grounds and we have not studied this in geography. 



%%%%%%%%%%%%%%%%%%%%%%%%%%%%%%%%%%%%%%%%%%%%%%%%%%
\section*{Data and Methods}
Complete documentation of the procedures and materials used in this study are available through the Survey of Reproducibility in Geographic Research Repository (\citet{Kedron_Holler_Bardin_Hilgendorf_2022} - \url{https://osf.io/5yeq8/}) hosted by the Open Science Framework (OSF). 
Before the start of data collection, we registered a preanalysis plan for the survey with OSF Registries (\citet{Kedron_Survey_PAP} - \url{https://osf.io/6zjcp}). 
We amended that plan at the close of data collection to reflect a change in stopping rule we used to end the survey. 

The survey was conducted under the approval and supervision of the Arizona State Institutional Review Board - \textit{STUDY00014232}.
All approved documentation, study protocols, and consent materials are available through the above repository.

%%%%%%%%%%%%%%%%%%%%%%%%
\subsection*{Survey Design}
Our target population of interest is researchers who have recently published in the field of geography. 
We followed a 4-step procedure to create a sampling frame for our survey that captures this diverse population of researchers and the approaches they use when studying geography. 

First, beginning at the publication level, we identified journals indexed as either geography or physical geography by the \href{https://access.clarivate.com/}{Web of Science's Journal Citation Reports} that also had a 5-year impact factor greater than 1.5.
From those journals, we created a database of all articles published between 2017 and 2021.  

Second, we used Arizona State University's institutional subscription to the \href{https://www.scopus.com/home.uri}{Scopus Database} to extract journal information (e.g., subject area, ranking), article information (e.g., abstract, citation counts), and author information (e.g., corresponding status, email) for each publication. 
Because our intention is to capture individuals actively publishing new geographic research, we retained publications indexed by Scopus as \textit{document type = "Article"} and removed all other publication types (e.g., editorials, book reviews) from our article database. 
We also removed articles with missing authorship information. 

Third, we next moved to the author level to create a condensed list of corresponding authors. 
We chose to focus on corresponding authors for two reasons. 
(1) Corresponding authorship is one indicator of the level of involvement an individual had in a given work. 
While imperfect, it was the best available indicator in the Scopus database as across journals there is no commonly adopted policy for declarations of author work (e.g., CRediT Statements).
(2) Scopus maintains email contact information for all corresponding authors, which gave us a means of contacting researchers in our sampling frame.
Scopus also maintains a unique identifier for each author (author-id) across time, which allowed us to identify authors across publications. 

Fourth, we associated each corresponding author with their contact email address and deduplicated to create a single record for each corresponding author. 
To maximize our chance or response, we retained the latest available contact information for each author by sorting the initial list by author-id and publication year (descending) and keeping the latest available entry for each author-id. 
For authors who had two or more distinct emails in the latest year of publication, we deduplicated by giving a ranked preference to .edu, .gov, and then .org extensions.

Applying these criteria yielded a sampling frame of 29,828 authors. 
On average, these authors published 2.7 articles in geography journals meeting our criteria between 2017 and 2021. 
Roughly one-third (33 percent) were most recently a corresponding author for an article published in a general geography journal. 
A similar proportion (32 percent) were most recently a corresponding author for an article published in an earth sciences journal, and smaller proportions in the social sciences and cultural geography (20 percent and 16 percent, respectively).

The survey instrument used in this study is available through the Survey of Reproducibility in Geographic Research Repository.
The survey consists of 23 questions that assess (i) perceptions of the reproducibility of geographic research, (ii) familiarity and use of reproducible research practices, and (iii) beliefs about barriers to reprodicibility. 
Survey questions were developed following a review of prior reproducibility surveys \citep[e.g.,][]{fanelli2009many,baker20161, konkol2019} and our own reading of recurring issues in the reproducibility literature. 
Because this survey had not been previously fielded, we pilot tested the instrument with a subset of \textit{n}=19 graduate students and geography faculty with differing levels of experience, topical focus, and methodological background. 
After pilot testing we removed these individuals from our sampling frame to ensure they would not be included in our final sample.

%%%%%%%%%%%%%%%%%%%%%%%%
\subsection*{Data Collection}
We used a digital form of the Tailored Design Method \citep{dillman2014internet} to survey geographic researchers between May 17 and June 10, 2022.
A simple random sample of 2,000 researchers was drawn without replacement from our sampling frame, and those researchers were invited via email to participate in the online survey. 
Researchers received their initial invitation on May 17, 2022. 
Two reminder emails were sent to researchers that had not yet completed the survey on May 26 and May 31, 2022.

Participation in the survey was entirely voluntary. 
Each researcher that opted to participate in the survey was provided with IRB approved consent documentation and linked to the internet survey instrument. 
Participants were also given the option to enter a random prize draw. 
The total possible compensation a participant was eligible to receive was 90 US dollars, awarded as a prepaid credit card.
Three prize winners were selected at random from those electing to enter the prize draw at the end of the data collection period.

The online survey was administered through \href{https://www.qualtrics.com/}{Qualtrics}. 
Each question on the survey was presented as a unique page. 
Adaptive questioning, conditionally displaying items based on responses to other items, was used to reduce the number of complexity of questions.
Participating researchers had the option to exit and re-enter the survey and were also able to review and/or change their answers using a back button as they progressed through the survey.

At the end of the data collection period, responses were checked for completeness and coded using the reporting standards of the American Association For Public Opinion Research \citep{aaporstandards}.
Responses were downloaded from Qualtrics, anonymized, and stored in a password-protected databases.
We used a unique key to link this data with participant information (e.g., citation history) stored in a separate database during our statistical analyses. 
To preserve participant privacy that connection was severed at the end of our analysis.
We have only shared the anonymized response file through our public repository. 
As a result some of the result presented below cannot be directly replicated. 
We have indicated throughout our results section which analysis can and cannot be replicated using the data files and code share through our public repository. 

%%%%%%%%%%%%%%%%%%%%%%%%
\subsection*{Statistical Analyses}


%%%%%%%%%%%%%%%%%%%%%%%%%%%%%%%%%%%%%%%%%%%%%%%%%%
\section*{Results}

A total of \textit{n}=215 of the authors we contacted completed the online survey with information sufficient for analysis. The contact rate for the survey was 13.9 percent, the response rate was 10.8 percent, yielding a cooperation rate of 77.6 percent. The refusal rate was 3.1 percent.\endnote{All outcome rates are reported using \citet{aaporstandards} standards. The outcome rates used were - response rate 2, cooperation rate 2, refusal rate 1, and contact rate 1.}   


%%%%%%%%%%%%%%%%%%%%%%%%%%%%%%%%%%%%%%%%%%%%%%%%%%
\section*{Discussion}


%%%%%%%%%%%%%%%%%%%%%%%%%%%%%%%%%%%%%%%%%%%%%%%%%%
\section*{Conclusion}
A standing question is how should geographic research approaches be designed to efficiently generate reliable knowledge.

\theendnotes


%%%%%%%%%%%%%%%%%%%%%%%%%%%%%%%%%%%%%%%%%%%%%%%%%%
\section*{Acknowledgement(s)}
We thank Tyler Hoffman for providing technical assistance in the development and execution of a set of trial queries using the Scopus API.

\section*{Funding}
This material is based on work supported by the National Science Foundation under Grant No. \textbf{BCS-2049837}.

\section*{Notes on contributor(s)}
\textbf{Kedron:} Conceptualization, Methodology, Writing - Original Draft, Writing - Review and Editing, Supervision, Project Administration, Funding Acquisition. \textbf{Holler:} Conceptualization, Methodology, Data Curation, Writing - Review and Editing, Funding Acquisition. \textbf{Bardin:} Conceptualization, Methodology, Writing - Original Draft, Writing - Review and Editing, Data Curation, Software.



\newpage
\bibliography{references}

\newpage
\noindent PETER KEDRON is an Associate Professor in the School of Geographical Science and Urban Planning and core faculty member in the Spatial Analysis Research Center (SPARC) at Arizona State University, Tempe, AZ, 85283, US. Email: Peter.Kedron@asu.edu. His research interests include spatial analysis, geographic information science, economic geography, and the accumulation of knowledge about geographic phenomena. \\  
  
\noindent JOSEPH HOLLER is an Assistant Professor of Geography at Middlebury College, Middlebury, VT, 05753, US. Email: \\
  
\noindent SARAH BARDIN is a PhD candidate ...

\end{document}
