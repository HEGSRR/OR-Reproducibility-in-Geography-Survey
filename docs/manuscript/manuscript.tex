\documentclass[]{interact}
\usepackage{epstopdf}% To incorporate .eps illustrations using PDFLaTeX, etc.
\usepackage{subfigure}% Support for small, `sub' figures and tables
%\usepackage[nolists,tablesfirst]{endfloat}% To `separate' figures and tables from text if required

\usepackage{natbib}% Citation support using natbib.sty
\bibpunct[, ]{(}{)}{;}{a}{}{,}% Citation support using natbib.sty
\renewcommand\bibfont{\fontsize{10}{12}\selectfont}% Bibliography support using natbib.sty

\theoremstyle{plain}% Theorem-like structures provided by amsthm.sty
\newtheorem{theorem}{Theorem}[section]
\newtheorem{lemma}[theorem]{Lemma}
\newtheorem{corollary}[theorem]{Corollary}
\newtheorem{proposition}[theorem]{Proposition}

\theoremstyle{definition}
\newtheorem{definition}[theorem]{Definition}
\newtheorem{example}[theorem]{Example}

\theoremstyle{remark}
\newtheorem{remark}{Remark}
\newtheorem{notation}{Notation}

\begin{document}

\articletype{DRAFT MANUSCRIPT}

\title{Reproducible Research Practices and Barriers to Reproducible Research in Geography: Insights from a Survey}

\author{
\name{Peter Kedron\textsuperscript{a,b}\thanks{CONTACT Peter Kedron. Email: Peter.Kedron@asu.edu}, Joseph Holler\textsuperscript{c}, and Sarah Bardin\textsuperscript{a,b}}
\affil{\textsuperscript{a}School of Geographical Sciences and Urban Planning, Arizona State University, Tempe, Arizona, USA; \textsuperscript{b}Spatial Analysis Research Center (SPARC), Arizona State University, Tempe, Arizona, USA; \textsuperscript{c}Department of Geography, Middlebury College, Middlebury, Vermont, USA}
}

\maketitle

\begin{abstract}
This template is for authors who are preparing a manuscript for a Taylor \& Francis journal using the \LaTeX\ document preparation system and the \texttt{interact} class file, which is available via selected journals' home pages on the Taylor \& Francis website.
\end{abstract}

\begin{keywords}
Reproducible Research, Epistemology, Geographic Research Methods
\end{keywords}

\maketitle

\section{Introduction}
Since the 1600s, replication has been a defining characteristic of the scientific method and an essential tool of researchers working to remove errors from our understanding of phenomena. 
\citet{nosek2020} broadly define a replication as any study that has at least one outcome that would be considered to be diagnostic evidence of a claim from prior research.
More frequently, replication is defined along two axes that help to distinguish the type of diagnostic evidence a study will provide and the function or purpose it is intended to serve. 
First, it is common to distinguish whether a replication study used the same data as the original study, or if new data were collected and analyzed. 
Second, it is helpful to identify whether a replication is focused on the question of whether the specific results of the original study can be reobserved, or whether the conclusions drawn from the original study are robust to changes in procedure or context.

When a researcher asks whether the same data and procedures can be used to generate the same results as an original study the central purpose of their study is verification.
If the researcher uses the original data, but introduces procedural differences they think may effect the original result they pursue a reanalysis designed to determine whether the original reasoning was somehow erroneous. 
Both of these approaches to replication assess the internal validity of research and are more commonly referred to as reproductions.
If the researcher tries to follow the procedures of an original study, but collects new data, the purpose shifts to evaluating the external validity of the original result by retesting it under new conditions.
This approach is commonly referred to as replication. 

While a replication or reproduction can never provide conclusive evidence for or against a finding, either type of study can be informative. 
If a well-executed, high-quality replication or reproduction recreates the result of an original study, we are apt to increase our confidence in the original findings. 
If a finding cannot be recreated, it reduces our confidence in the original result and suggests that our current understanding of the system being studied or our methods of testing that system are insufficient.

\section{Prior Survey Research on Reproducible Research}

\end{document}
